Ultrasound imaging has been widely used in bio-medical imaging diagnosis for a long history because of its merits: no radiation, high penetration depth, and real-time imaging capability. In this paper, we propose an ultrasound-based system that monitors respiratory status of asthma subjects via detecting of diaphragm movement. This system implements Chan-Vese algorithm to accurately segment diaphragm area from ultrasound image sequences and extracts 1D breathing waveform by computing mutual information (MI) between two consecutive ultrasound frames. In addition, four types of respiratory signals are identified: normal breath, fast breath, apnoea, and cough, which are related to four symptoms of asthma attack and defined as the breathing templates used for early asthma detection. In experiments, the proposed system is evaluated with a public dataset from 'Ultrasound image gallery' which contains 9 ultrasound videos and our dataset collected by 'Interson Seemore' probe which contains 5 ultrasound videos in the diaphragm area. The results show that Chan-Vese segmentation method is superior to the other three algorithms: adaptive thresholding, EM/MPM, and Fuzzy C Means (FCM), and MI is a feasible method to extract accurate respiratory signal and clear information of the phase of respiratory cycle from 2D images.
